\documentclass[a4paper]{jsarticle}
\usepackage[dvipdfmx]{graphicx}
\usepackage{braket}

% 数式を(Chapter.Number)の形にする。
\makeatletter
\renewcommand{\theequation}{  
  \thesection.\arabic{equation} }
  \@addtoreset{equation}{section}
\makeatother

\def\vector#1{\mbox{\boldmath $#1$}}
\def\braces#1{\left( #1 \right)}
\def\bracem#1{\left\{ #1 \right\}}
\def\braceb#1{\left[ #1 \right]}
\def\lefts{\left(}
\def\rights{\right)}
\def\leftb{\left[}
\def\rightb{\right]}
\def\leftm{\left\{}
\def\rightm{\right\}}
\begin{document}
\title{FEM-DVRのまとめ}
\author{松崎 黎}
\maketitle

\section{まとめ}
see T.Rescigno and C.W.McCurdy, PRA *62*, 032706 (2000)
Lobatto shape関数を用いる。
\begin{eqnarray}
  f_{i,m}(x) = \Pi_^{j!=m} \frac{x-x_j^i}{x_m^i-x_j^i} \ \ \ (r_i<x<r_{i+1})
\end{eqnarray}
$i$番目の有限要素にある$m$番目のLS基底を表現している。用意に証明できるように
次の性質を満たす。
\begin{eqnarray}
  f_{i,m}(x^{i'}_{m'}) = delta(i,i')\delta(m,m')
\end{eqnarray}
を得る。あとで使うために、一階微分を計算すると、
\begin{eqnarray}
  \frac{d}{dr} f_i(x_j) = \frac{1}{x_i-x_j} \Pi_{k!=i,j} \frac{x_j-x_k}{x_i-x_k} (i!=j) \\
  \frac{d}{dr} f'_{im}(x_{im'}) = \frac{1}{2w^i_m}(\delta_{m,n} - \delta_{m,1})
\end{eqnarray}
となる。行列要素を計算すると、
\begin{eqnarray}
  \braket{f_{im} | V | f_{i'm'}} = \delta_{ii'}\delta_{mm'}V(x^i_m)w^i_m
\end{eqnarray}
となる。ベクトルは、
\begin{eqnarray}
  \braket{f_{im} | u} = w^i_m u(x^i_m)
\end{eqnarray}
で与えられる。一階微分は、
\begin{eqnarray}
  \braket{f_{im} | f'_{i'm'}} = w^i_m f'_{im'}(x^i_m) delta(ii')
\end{eqnarray}
となる。

次に規格化されたDVR基底を用意する。

この関数は用意にわかるように、

を満たすので、重なり積分およびポテンシャルの積分は、
\begin{eqnarray}
  \braket{f_i | f_j} = \delta_{ij}\\
    \braket{f_i |V| f_j} = V(x_i)\delta_{ij}\\
\end{eqnarray}


次に、それぞれの有限区間をまたぐ基底を用意する。
\begin{eqnarray}
  \chi_i(r) = \braces{\sqrt{w_n^if^i_n(r) v \sqrt{w_1^{i+1}f^{i+1}_1(r)}}}/sqrt{w^i_n+w^{i+1}_1}
\end{eqnarray}
これらを合わせて、基底関数セット$\phi_i$とおくと、
\begin{eqnarray}
  \braket{\phi_i |V| \phi_j} = \delta_{ij}V(r_i)
\end{eqnarray}
を得る。

\bibliographystyle{jplain}
\bibliography{library,books,paper}
\end{document}